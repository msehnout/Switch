\documentclass[11pt,a4paper,titlepage]{article}
\usepackage[a4paper]{geometry}
\usepackage[utf8]{inputenc}
\usepackage[english]{babel}
\usepackage{lipsum}

\usepackage{amsmath, amssymb, amsfonts, amsthm, fouriernc, mathtools}
% mathtools for: Aboxed (put box on last equation in align envirenment)
\usepackage{microtype} %improves the spacing between words and letters

\usepackage{graphicx}
\graphicspath{ {./pics/} {./eps/}}
\usepackage{epsfig}
\usepackage{epstopdf}

%% Source code:
\usepackage{minted} 
\newminted{bash}{fontsize=\footnotesize}


%%%%%%%%%%%%%%%%%%%%%%%%%%%%%%%%%%%%%%%%%%%%%%%%%%
%% COLOR DEFINITIONS
%%%%%%%%%%%%%%%%%%%%%%%%%%%%%%%%%%%%%%%%%%%%%%%%%%
\usepackage{xcolor} % Enabling mixing colors and color's call by 'svgnames'
%%%%%%%%%%%%%%%%%%%%%%%%%%%%%%%%%%%%%%%%%%%%%%%%%%
\definecolor{MyColor1}{rgb}{0.2,0.4,0.6} %mix personal color
\newcommand{\textb}{\color{Black} \usefont{OT1}{lmss}{m}{n}}
\newcommand{\blue}{\color{MyColor1} \usefont{OT1}{lmss}{m}{n}}
\newcommand{\blueb}{\color{MyColor1} \usefont{OT1}{lmss}{b}{n}}
\newcommand{\red}{\color{LightCoral} \usefont{OT1}{lmss}{m}{n}}
\newcommand{\green}{\color{Turquoise} \usefont{OT1}{lmss}{m}{n}}
%%%%%%%%%%%%%%%%%%%%%%%%%%%%%%%%%%%%%%%%%%%%%%%%%%




%%%%%%%%%%%%%%%%%%%%%%%%%%%%%%%%%%%%%%%%%%%%%%%%%%
%% FONTS AND COLORS
%%%%%%%%%%%%%%%%%%%%%%%%%%%%%%%%%%%%%%%%%%%%%%%%%%
%    SECTIONS
%%%%%%%%%%%%%%%%%%%%%%%%%%%%%%%%%%%%%%%%%%%%%%%%%%
\usepackage{titlesec}
\usepackage{sectsty}
%%%%%%%%%%%%%%%%%%%%%%%%
%set section/subsections HEADINGS font and color
\sectionfont{\color{MyColor1}}  % sets colour of sections
\subsectionfont{\color{MyColor1}}  % sets colour of sections

%set section enumerator to arabic number (see footnotes markings alternatives)
\renewcommand\thesection{\arabic{section}.} %define sections numbering
\renewcommand\thesubsection{\thesection\arabic{subsection}} %subsec.num.

%define new section style
\newcommand{\mysection}{
\titleformat{\section} [runin] {\usefont{OT1}{lmss}{b}{n}\color{MyColor1}} 
{\thesection} {3pt} {} } 

%%%%%%%%%%%%%%%%%%%%%%%%%%%%%%%%%%%%%%%%%%%%%%%%%%
%       CAPTIONS
%%%%%%%%%%%%%%%%%%%%%%%%%%%%%%%%%%%%%%%%%%%%%%%%%%
\usepackage{caption}
\usepackage{subcaption}
%%%%%%%%%%%%%%%%%%%%%%%%

%%%%%%%%%%%%%%%%%%%%%%%%%%%%%%%%%%%%%%%%%%%%%%%%%%
%       !!!EQUATION (ARRAY) --> USING ALIGN INSTEAD
%%%%%%%%%%%%%%%%%%%%%%%%%%%%%%%%%%%%%%%%%%%%%%%%%%
%using amsmath package to redefine eq. numeration (1.1, 1.2, ...) 
%%%%%%%%%%%%%%%%%%%%%%%%
\renewcommand{\theequation}{\thesection\arabic{equation}}

%set box background to grey in align environment 
\usepackage{etoolbox}% http://ctan.org/pkg/etoolbox
\makeatletter
\patchcmd{\@Aboxed}{\boxed{#1#2}}{\colorbox{black!15}{$#1#2$}}{}{}%
\patchcmd{\@boxed}{\boxed{#1#2}}{\colorbox{black!15}{$#1#2$}}{}{}%
\makeatother
%%%%%%%%%%%%%%%%%%%%%%%%%%%%%%%%%%%%%%%%%%%%%%%%%%

\makeatletter
\let\reftagform@=\tagform@
\def\tagform@#1{\maketag@@@{(\ignorespaces\textcolor{red}{#1}\unskip\@@italiccorr)}}
\renewcommand{\eqref}[1]{\textup{\reftagform@{\ref{#1}}}}
\makeatother
\usepackage{hyperref}
\hypersetup{colorlinks=true}


%%%%%%%%%%%%%%%%%%%%%%%%%%%%%%%%%%%%%%%%%%%%%%%%%%
%% PREPARE TITLE
%%%%%%%%%%%%%%%%%%%%%%%%%%%%%%%%%%%%%%%%%%%%%%%%%%
\title{\blue Computer Networking I \\
\blueb Switch assignment}
\author{Martin Sehnoutka}
\date{\today}
%%%%%%%%%%%%%%%%%%%%%%%%%%%%%%%%%%%%%%%%%%%%%%%%%%



\begin{document}
\maketitle

\tableofcontents

\section{Messages format}

\section{Software architecture}

\subsection{Server side - switch}

\begin{minted}{haskell}
main :: IO()
main = do
  controlChannel <- atomically newTChan
  readChannel <- atomically newTChan
  let channels = (controlChannel, readChannel)
  socket <- listenOn $ PortNumber 4242
  putStrLn $ Tg.control ++ "Starting switch on port: " ++ show 4242
  forkIO $ switch channels []
  acceptLoop socket channels
\end{minted}

\subsection{Client side}

\section{Installation on Lintula machines}

First step is to add ghc \footnote{Glasgow Haskell Compiler} and cabal \footnote{Common Architecture for Building Applications and Libraries} into your path variable:
\begin{bashcode}
$ PATH=$PATH:/usr/local/lang/haskell/ghc-7.10.2/bin
\end{bashcode}
Then it is necessary to update cabal package database. These packages will be installed into \textasciitilde/.cabal directory, thus it does not need root privileges. 
\begin{bashcode}
$ cabal update
\end{bashcode}
Now it is possible to install both switch and client using cabal installers.
\begin{bashcode}
$ cd <SWITCH_DIR>/switch
$ cabal sandbox init
$ cabal install -j
$ cd <SWITCH_DIR>/client
$ cabal sandbox init
$ cabal install -j
\end{bashcode}

\section{Testing}
I wrote little script for testing purposes. This script runs switch and then in loop starts 8 clients. Output of each process is stored in different file: switch.log, client0.log, client1.log, etc.. By exploring these file I can compare messages that arrived to each client and switch.

For example I can explore messages that arrived into switch with destination address 3 and compare them with client 3 output:
\begin{bashcode}
$ cat switch.log | grep --regexp='(Unicast) [0-9]->3' | tail --lines=6
[client]  18:16:52 (Unicast) 5->3: uksvykjm
[client]  18:16:58 (Unicast) 1->3: kgxqmenyqw
[client]  18:17:03 (Unicast) 2->3: vntppmmkjudrwag
[client]  18:17:05 (Unicast) 0->3: uuonx
[client]  18:17:05 (Unicast) 7->3: turvwhxhf
[client]  18:17:05 (Unicast) 6->3: plexvthznugwllv
$ cat client3.log | grep --regexp='(For me)' | tail --lines=6
(For me) 1->3: kgxqmenyqw
(For me) 2->3: vntppmmkjudrwag
(For me) 0->3: uuonx
(For me) 7->3: turvwhxhf
(For me) 6->3: plexvthznugwllv
(For me) 7->3: nuwzqpmkko
\end{bashcode}
As you can see, messages are sent and received in different order, but this is expected behavior. I can do the same for broadcast messages:
\begin{bashcode}
$ cat switch.log | grep Broadcast | head --lines=5
[client]  18:16:45 (Broadcast) 7->8: mdowboqvzxjg
[client]  18:16:46 (Broadcast) 3->10: ywspkekbq
[client]  18:16:46 (Broadcast) 4->8: ofxcfrixmtqps
[client]  18:16:46 (Broadcast) 6->8: xqgtta
[client]  18:16:46 (Broadcast) 5->8: zfhkjya
$ cat client7.log | grep Broadcast | head --lines=5
(Broadcast) 3->10: ywspkekbq
(Broadcast) 4->8: ofxcfrixmtqps
(Broadcast) 6->8: xqgtta
(Broadcast) 5->8: zfhkjya
(Broadcast) 5->9: bclpwbfpshnuum
\end{bashcode}
In this case I can even see, that switch does not send messages back to their source address because very first message is from client 7, but is not present in client's log file.

\end{document}
